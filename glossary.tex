\newglossaryentry{Stereotyp}
{
  name=Stereotyp,
  description={
    Ein Stereotyp ist eine Erweiterung vorhandener Modellelemente in der Unified Modeling Language (UML). In der Praxis geben Stereotype vor allem die möglichen Verwendungszusammenhänge (Verwendungskontext) einer Klasse, einer Beziehung oder eines Paketes an}
}

\newglossaryentry{Requirement} {
name=Requirement,
plural=Requirements,
description={
In der (Software-)Technik ist eine Anforderung (häufig englisch requirement) eine Aussage über eine zu erfüllende Eigenschaft oder zu erbringende Leistung eines Produktes, Systems oder Prozesses.}
}

\newglossaryentry{Spezifikation}
{
  name=Spezifikation,
  description={
    Eine Spezifikation ist die Beschreibung eines Produktes, eines Systems oder einer Dienstleistung durch Auflistung seiner Anforderungen}
}

\newglossaryentry{Sandbox}
{
  name=Sandbox,
  description={
    Ein dedizierter "Software-Container" in dem eine App läuft. Die App hat nur Zugriffsrechte auf den Speicherbereich der Sandbox und kann somit keinen Schaden außerhalb der Sandbox anrichten. Die Verwendung von Sandboxes ist ein Sicherheits- und Datenschutz-Feature}
}

\newglossaryentry{Usability}
{
name=Usability,
description={
  Gebrauchstauglichkeit (englisch usability) bezeichnet das Ausmaß, in dem ein Produkt, System oder ein Dienst durch bestimmte Benutzer in einem bestimmten Anwendungskontext genutzt werden kann, um bestimmte Ziele effektiv, effizient und zufriedenstellend zu erreichen}
}

\newglossaryentry{Ergonomie} {
name=Ergonomie,
description={
  Ziel der Ergonomie ist es, handhabbare und komfortabel zu nutzende Produkte herzustellen}
}

\newglossaryentry{Package} {
name=Package,
plural=Packages,
description={
Ein Paket (englisch package) fasst eine Menge von Modellelementen zu einer Gruppe zusammen und bildet einen Namensraum für sie}
}

\newglossaryentry{Notification} {
name=Notification,
plural=Notifications,
description={
Eine Notification (dt. Benachrichtigung) ist eine Pop-up-Meldung, die den Benutzer über ein bestimmtes Ereignis oder einen Vorgang informiert.
}
}
