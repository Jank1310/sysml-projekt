\section{State Diagrams}
Die \textbf{State Diagram} in \textbf{Phase 2} werden dazu benutzt, um das interne Verhalten der \textit{SmartWatch} bei ausgewählten Benutzerinteraktionen darzustellen. Die Diagramme definieren, das Verhalten und die Bearbeitung bei einzelnen Eingaben. Zu diesem Zeitpunkt der Entwicklungsphase wurden die Zustände für die \textit{Verbindung zum Smartphone} \textit{(Pairing)}, das Verwenden der \textit{Fitnessapplikationen} und deren \textit{Speicherfunktion}, sowie das Verhalten bei \textit{ein- und ausgehenden Anrufen} modelliert.\\
\begin{figure}[h]
\centering\
\includegraphics[width=\textwidth]{img/statePairing}
\caption[State Diagram: Pairing]{Verhalten der SmartWatch beim Pairing mit dem SmartPhone.}
\label{fig:statePairing}
\end{figure}
Wenn sich die \textit{SmartWatch} mit dem \textit{SmartPhone} verbinden will, wird zunächst beim Handy mittels \textit{Bluetooth} nach der \textit{SmartWatch} gesucht. Dabei muss bei \textbf{beiden} Geräten das \textit{Bluetooth-Modul} aktiviert sein, da sonst keine Verbindung möglich ist. Wenn sich die Geräte gegenseitig gefunden haben, wird überprüft ob sich auf dem \textit{SmartPhone} die zur \textit{SmartWatch} dazugehörige \textit{Companion-App} befindet. Sollte diese noch nicht installiert sein, wird der Benutzer dazu aufgefordert dies zu tun. Ohne die installierte \textit{Companion-App} ist ein \textit{Pairing} nicht möglich. Wenn die App funktionstüchtig ist, wird Benutzer aufgefordert den vom \textit{SmartPhone} generierten Code auf der \textit{SmartWatch} einzugeben.Dies dient dazu, dass nicht willkürliche \textit{SmartPhones} sich mit der \textit{SmartWatch} verbinden können. Wenn der Code erfolgreich eingegeben wurde, erfolgt die Verbindung zwischen beiden Geräten. Bei jedem der Schritte zum Verbindungsaufbau ist es möglich, den Vorgang abzubrechen und \textit{SmartPhone} und \textit{SmartWatch} in den Ruhenzustand zurück zu versetzen Abb. \ref{fig:statePairing}.\\
\begin{figure}[h]
\centering\
\includegraphics[width=\textwidth]{img/stateFitness}
\caption[State Diagram: Fitness]{Verhalten der SmartWatch beim starten der Fitnessapp.}
\label{fig:stateFitness}
\end{figure}
Wenn die \textit{native Fitnessfunktionalität} der \textit{SmartWatch} benutzt werden will, muss die Uhr zunächst aus dem Ruhezustand geholt werden. Anschließend wird über das Menü die \textit{Fitnessapp} ausgewählt und bestätigt. Anschließend wird die Routine zur Aktivitätsaufzeichnung ausgeführt. Sobald diese beendet ist, versucht die \textit{SmartWatch} die Fitnessdaten im \textit{internen Speicher} abzulegen. Vorausgesetzt der \textit{Speichervorgang} war erfolgreich, werden die Daten auf dem internen Speicher abgelegt. Bei einem Fehler während des Speicherns, werden die Daten gelöscht und ein Fehlerprotokoll erstellt Abb. \ref{fig:stateFitness}.\\
\begin{figure}[h]
\centering\
\includegraphics[width=\textwidth]{img/stateSync}
\caption[State Diagram: Synchronisation]{Verhalten der SmartWatch beim Synchronisieren der Fitnessaktivitäten mit dem SmartPhone.}
\label{fig:stateSync}
\end{figure}
Nachdem eine \textit{Fitnessaktivität} beendet wurde und die \textit{Fitnessdaten} auf der \textit{SmartWatch} zwischengelagert wurden, wird versucht bei der nächsten Verbindung mit dem \textit{SmartPhone}, die \textit{Daten} von der \textit{Uhr} auf das \textit{SmartPhone} zu übertragen Abb. \ref{fig:stateSync}. Voraussetzung dafür ist eine erfolgreiche \textit{Verbindung} zwischen den beiden Geräten. Sobald die \textit{Verbindung} steht, wird mit dem \textit{Synchronisationsvorgang} begonnen. Bei diesem werden zunächst die \textit{Daten} von der \textit{Uhr} auf das \textit{Telefon} gespeichert, um einen möglichen Datenverlust zu vermeiden. Anschließend werden die \textit{Daten} von der \textit{SmartWatch} gelöscht und die \textit{Synchronisation} beendet. Sollte im Verlauf der \textit{Synchronisation} ein Fehler auftreten, wird die \textit{Synchronisation} abgebrochen und ein \textit{Fehlerprotokoll} erstellt. \\
\begin{figure}[h]
\centering\
\includegraphics[width=\textwidth]{img/stateAnrufEingehend}
\caption[State Diagram: eingehender Anruf]{Verhalten der SmartWatch bei einen eingehenden Anruf.}
\label{fig:stateAnrufEingehend}
\end{figure}
Als nächstes wird das Verhalten der \textit{SmartWatch} bei einem \textit{eingehenden Anrufen} näher betrachtetAbb. \ref{fig:stateAnrufEingehend}. Zunächst befinden sich sowohl \textit{Handy} als auch \textit{Uhr} im \textit{Ruhezustand}. Sobald ein \textit{Anruf} eingeht, wechselt zuerst das \textit{SmartPhone} in den \textit{Anrufzustand} und gibt ein Signal an die \textit{Uhr} weiter. Sobald die \textit{Uhr} das Signal empfängt, wechselt auch sie in einen \textit{Anrufzustand}. In diesem Zustand, wird die Bildschirmbeleuchtung aktiviert und die \textit{Anruferanzeige} eingeblendet. Diese \textit{Anzeige} ermöglicht es dem Benutzer zu wählen ob er den Anruf entgegen nehmen will oder nicht. Sollte er den Anruf annehmen, wechselt die \textit{SmartWatch} in den Zustand \textit{Telefonieren} und aktiviert die das interne \textit{Mikrophon}. Sobald der Anruf beendet wurde, kehren \textit{Telefon} und \textit{Uhr} in den \textit{Ruhezustand} zurück. \\
\begin{figure}[h]
\centering\
\includegraphics[width=\textwidth]{img/stateAnrufAusgehend}
\caption[State Diagram: ausgehender Anruf]{Verhalten der SmartWatch beim Tätigen eines Anrufs.}
\label{fig:stateAnrufAusgehend}
\end{figure}
Die von uns entwickelte \textit{SmartWatch} soll nicht nur in der Lage sein, Anrufe anzunehmen sondern auch fähig sein \textit{Anrufe zu tätigen} Abb. \ref{fig:stateAnrufAusgehend}. Auch hier wird wieder davon ausgegangen, dass sich beide Geräte im \textit{Ruhezustand} befinden. Beendet wird dieser durch das Aufrufen der \textit{Telefonbuchanzeige} auf der \textit{SmartWatch}. Sobald der Benutzer einen Kontakt über die \textit{SmartWatch}aus dem Telefonbuch des \textit{SmartPhones} gewählt hat, wird die \textit{Anruffunktionalität} eingeleitet. Sollte der Angerufene das Telefonat annehmen wird auch hier in den Zustand \textit{Telefonieren} gewechselt. Dieser Zustand verhält sich genauso wie bei einem \textit{eingehenden Anruf}. Nach Beendigung des Gesprächs wechseln beide Geräte wieder in den \textit{Ruhezustand}.