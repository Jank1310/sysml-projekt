\chapter{Fazit}

Der Softwareentwurf einer Smartwatch erfordert eine gewisse Kompromissbereitschaft. Im Falle eines portablen Geräts, welches zusätzlich am Körper getragen werden soll, sind viele Ideen mit der Realität nicht leicht vereinbar. Insbesondere die Hardware betreffend. Viele initiale Ideen (z.B. \gls{native} Apps für Nicht-Fitness Anwendungen) wurden nach längerer Überlegung verworfen, weil der gewünscht kleine Formfaktor keinen räumlichen Platz (z.B. für viele Sensoren, Akku im Gehäuse) bieten kann. Daher musste im gesamten Projekt der Fokus auf ein paar wenigen Kernfunktionalitäten der Uhr liegen: Uhrzeit und Datum, Mitteilungen und Fitness-Funktionen. Die Zielgruppe des Geräts sind technikaffine Menschen mit einem Hang zu sportlichen Aktivitäten und genau diese Zielgruppe soll vom Ergebnis des Projekts profitieren.
