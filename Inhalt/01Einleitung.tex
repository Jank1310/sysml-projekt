\chapter{Einleitung}

\section{Motivation}
Matthias' Text hier...\\
Den Absatzzahlen von 2014 nach zu Urteilen, ist der Markt für Smartwatches stetig am steigen !!!Referenz!!!. Durch die Vielzahl an Anbietern entsteht hierbei auch eine große Anzahl verschiedener Modelle, die auf verschiedene Benutzergruppen ausgelegt sind. Die Zielgruppe der von uns entwickelten Smartwatch ist primär die Gruppe der durchschnittlichen Benutzer, welche die Uhr primär als Kommunikationshilfsmittel und Sportequipment benutzen wollen. Aufgrund dessen, wurden die Funktionalitäten !!!Anhang TagCloud!!! vorwiegend auf diese Benutzergruppe zugeschnitten.

\section{Stand der Technik}
Markus' Text hier...

\section{Phasenbeschreibung}
Das folgende Dokument beschreibt die einzelnen Phasen des Projekts:
\begin{itemize}
  \item Phase 1: Spezifikation -- In der Spezifikations-Phase wird das Grob-Design sowie der Verteilung und das Deployment des Projekts beschrieben. In dieser Phase werden die Requirements, die von der Smartwatch erfüllt werden sollen, aufgelistet und beschrieben. Zusätzlich wird eine erste Aufteilung der Projektelemente in Packages vorgenommen und erste Designs der Hardware in Form von Block-Definition-Diagrams entworfen. Erste Designentscheidungen werden ebenfalls getroffen.

  \item Phase 2: System-Interfaces und System-Interaktion -- In dieser Phase wird die Bedienung des Systems sowie die Interaktion der Smartwatch mit Smartphones entworfen. Des Weiteren werden die Use-Cases verfeinert und erweitert. Die in Phase 1 getroffenen Designentscheidungen werden geprüft und überarbeitet, um neuste Erkenntnisse zum Projekt einfließen zu lassen.

  \item Phase 3: Fein-Design -- In dieser Phase wird die Interaktion zwischen Benutzer und Smartwatch entworfen und detailliert beschrieben. 
\end{itemize}
