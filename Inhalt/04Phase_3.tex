\chapter{Phase 3: Fein-Design und Installation/Konfiguration}

\section{Package Diagram}
Das Smartwatch besitzt einige interessante Funktionen: einen präzisen Touchscreen, Apps und Widgets, etwa zum Lesen von Mails, SMS, Twitter und Facebook sowie Infos vom gekoppelten Smartphone, etwa über verpasste Anrufe und den Akkustatus.
Standard Anwendungen die mit dem Smartwatch geliefert werden, sind Uhrzeitangaben, Erinnerungsfunktionen, Kalender-Funktionalität, oder die Aktivitätserkennung.
Klassische Anwendungen sind auch die Fitnessfunktionalitäten.
Mit der mitegelieferten API \textit{AppKit} lassen sich Apps für die SmartWatch programmieren, die eng mit dem Smartpohone zusammenarbeiten.
Die softwaretechnische Paketstruktur ist im Package \textit{Software} abgebildet.

\begin{figure}[h]
\centering\
\includegraphics[width=\textwidth]{img/PackagePhase2}
\caption{Packagediagram}\label{fig:package}
\end{figure}

\section{Class Diagram}

\begin{figure}[h]
\centering\
\includegraphics[width=\textwidth]{img/classdiagram}
\caption{Relevante Software-Komponenten des Smartwatch-Systems in einem Klassendiagramm}\label{fig:class}
\end{figure}


\section{Block Definition Diagram}
In Phase 3 stellt das Block Definition Diagram wieder den physikalischen Aufbau der \textit{SmartWatch} dar (Abb. \ref{fig:block2}). Da nun alle Funktionalitäten fertig definiert wurden, ist dies der endgültige Aufbau der Uhr. Die zwei Hauptkomponenten der \textit{SmartWatch} sind weiterhin das \textit{SmartWatch-Modul} und das \textit{Armband}, wobei das \textit{Modul} die Funktionalitäten und die Software verwaltet und das \textit{Armband} die Uhr mit Strom versorgt. Im Zentrum des Diagramms ist das \textit{SmartWatch-Modul}, welches eine \textit{Bluetooth-Antenne} besitzt. Über diese Schnittstelle kommuniziert die \textit{Uhr} mit einem gekoppelten \textit{SmartPhone}. Das \textit{Modul} besitzt einen \textit{Display}, dessen Auflösung in Pixel angegeben wird und welches über \textit{drei} \textit{Tasten} bedient wird. Eine Taste ist das \textit{Zoomrad}. Damit wird die Schriftgröße angezeigter Nachrichten oder laufender Applikationen gesteuert. Zusätzlich kann mit den \textit{Scrolltasten} die \textit{SmartWatch} aus dem Ruhezustand geholt oder nach dem Herunterfahren wieder gestartet werden. Eine weitere Komponente ist der interne \textit{Speicher} zum Verwalten der nativen Fitnessapplikationen und Fehlerprotokollen. Außerdem hält das \textit{SmartWatch-Modul} einen \textit{Vibrationsmotor}. Auf diesen \textit{Motor} können Applikationen vom \textit{Smartphone} oder auch die nativen Fitnessfunktionen zugreifen, um den Benutzer über spezielle Ereignisse zu informieren. Für die Tonausgabe bei Verwendung einer Musikapplikation oder bei einem Telefonat ist ein \textit{Lautsprecher} vorhanden. Dieser gibt jede Form von Audiosignal, die sonst über die \textit{SmartPhone-Lautsprecher} ausgegeben werden würde, stattdessen über die \textit{SmartWatch} aus. Das integrierte \textit{Mikrophon} dient als Kommunikationsschnittstelle für Telefonate. Das aufgefangene Audiosignal wird an das \textit{Smartphone} weitergeleitet und dient als Ersatz für das \textit{Mikrophon} im \textit{SmartPhone}. Für die Verwendung durch die nativen Fitnessapplikationen und auch für den Zugriff durch andere SmartPhone-Applikationen stehen ein \textit{GPS-Modul} und \textit{Sensoren} zur Verfügung. Da die Fitness-Funktionalität auch ohne \textit{SmartPhone} ausführbar sein sollten, können dafür die vorhandenen Modalitäten des \textit{SmartPhones} nicht verwendet werden. Als \textit{Sensoren} stellt die \textit{SmartWatch} ein \textit{Gyroskop}, ein \textit{Temperatursensor} und ein \textit{Pulsmesser}. Das \textit{Gyroskop} ermöglicht es, in Verbindung mit dem \textit{GPS}, zurückgelegte Strecken zu messen. Der \textit{Temperatursensor} misst die aktuelle Körpertemperatur über die Haut des Benutzers. Der letzte \textit{Sensor} ist der \textit{Pulsmesser}. Dieser misst den Puls des Benutzer über das Handgelenk.\\
Als nächste Hauptkomponente wird das \textit{Armband} betrachtet. Das \textit{Armband} soll die Stromversorgung der \textit{SmartWatch} regeln und ist austauschbar. So gibt es verschiedene Modelle für Fitness und Alltag. Diese aus einer unterschiedlichen Anzahl von \textit{Akkuzellen} und bestehen deshalb aus verschiedenen Materialien, aus denen wiederum ein Unterschied in Gewicht und Komfort der \textit{Armbänder} folgt.
\begin{figure}[h]
\centering\
\includegraphics[width=\textwidth]{img/block2}
\caption{Endgültiger physikalischer Aufbau der SmartWatch in Phase 3 anhand eines Block Definition Diagram.}\label{fig:block2}
\end{figure}

\section{Timing Diagram}
Die Anforderungen aus den \glspl{Use-Case} \textbf{Boot.01} und \textbf{Poweroff.01}, die sich auf das Ein- und Ausschalten der Smartwatch beziehen, werden in den Abbildungen~\ref{fig:timing_diagram_power_on} und ~\ref{fig:timing_diagram_power_off} als Timing Diagrams dargestellt. Aus den \glspl{Requirement} geht hervor, dass sowohl das Einschalten als auch das Herunterfahren maximal 10 Sekunden dauern darf. Die Zeitmessung beginnt nachdem der entsprechende Vorgang durch das lange Drücken einer beliebigen Taste gestartet wurde.

\begin{figure}
\centering\
\includegraphics[width=14cm]{img/timing_diagram_power_on}
\caption[Timing Diagram: Power-On]{Darstellung des Startvorgangs der Smartwatch. Zur Verdeutlichung wird die \textit{Tasten}-Komponente seperat dargestellt}
\label{fig:timing_diagram_power_on}
\end{figure}

Um den einsekündigen Tastendruck zum Starten und den 5-sekündigen zum Ausschalten der Smartwatch zu verdeutlichen, wurde die \textit{Tasten}-Komponente in den Diagrammen separat dargestellt, obwohl sie eigentlich ein Bestandteil von \textit{Smartwatch-Modul} ist (siehe Abb.~\ref{fig:block2}).

\begin{figure}
\centering\
\includegraphics[width=14cm]{img/timing_diagram_power_off}
\caption[Timing Diagram: Power-Off]{Timing Diagram zur Verdeutlichung des Herunterfahrens der Smartwatch}
\label{fig:timing_diagram_power_off}
\end{figure}


\section{Parametric Diagram}
Im folgenden wird die Anforderung an die Akkulaufzeit ausführlich als Parametric Diagram dargestellt und erläutert.
Da der Verbrauch des Smartwatches von großem Interesse ist, wird der Verbrauch
mit Analytische Batteriemodelle modelliert.
Außerdem ist in der Anforderung des Smartwatches festgelegt, das die Akkulaufzeit bei normaler Benutzung mindestens 24 Stunden halten muss. %% ref???
Der Verbrauch eines Systems ist die Summe der Verbrauchswerte einzelner Subkomponenten, wie z.B die Sensoren und Benutzersoftware. Die Subkomponenten dienen als Input für die Laufzeitmodellierung des Smartwatches.
Die Constraint Blöcke dienen der graphischen Repräsentation von Bedingungen für die Ladezeit, die für das Smartwatch verwendet werden sollen.
Analytische Batteriemodelle sind die erste Klasse der Batteriemodelle, welche
hier näher untersucht werden.
Diese beschreiben mathematisch durch eine oder mehrere analytische Gleichungen den Verbrauch einer Batterie.
Eines der ersten und einfachsten analytischen Batteriemodelle ist die Formel
von Peukert, welche die Entladezeit oder auch tatsächlich nutzbare Kapazität
einer Batterie unter einer konstanten Last berechnen kann.
Die Formel von Peukert ist wie folgt \cite{peukert}:
\[
T= \frac{C}{I^{n}}
\]
Die Kapazität \textit{C} ist in Ampere h gegeben.
\textit{I} steht fur den Strom (in Ampere). Der Parameter \textit{n} steht fur die
Peukert-Zahl von der verwendeten Batterie.\\
Weiterhin gibt es stochastische Batteriemodelle, dass das Entladeverhalten einer Batterie
über einen stochastischen Prozess modeliert.
Das Modell von Panigrahi ist ein stochastisches Modell.
Im Modell wird zwischen der theoretischen Kapazität \textit{T} und der tatsächlich verfugbaren
Kapazität \textit{N} der Batteriezelle unterschieden.
Die Formel ist in der Constraint-Block abgebildet (siehe Abbildung~\ref{fig:blockBattery}) und in der Literatur näher beschrieben \cite{pan}.
In der Abbildung~\ref{fig:blockBattery} ist ein Block Definition Diagram angefertigt, welche die einzelnen Constraint Blöcke beinhaltet und ihren inneren Aufbau darstellt.

\begin{figure}[h]
\centering\
\includegraphics[width=14cm]{img/batterybdd}
\caption{Block Definition Diagram mit den einzelnen Constraint Blöcken}
\label{fig:blockBattery}
\end{figure}

Die Verbindungen der einzelnen Systemeigenschaften mit den Constraint-Parameter ist in
Abbildung~\ref{fig:parBattery} dargestellt.
In diesem Diagramm kann man die Beziehungen zwischen den Parametern der einzelnen Constraints und den Systemeigenschaften, hier sind es die Sensoren sehen.

\begin{figure}[h]
\centering\
\includegraphics[width=14cm]{img/batteryParametric}
\caption{Parametric Diagram für die Conditions Akkulaufzeit}
\label{fig:parBattery}
\end{figure}


\section{Activity Diagram}
\begin{figure}[h]
\centering\
\includegraphics[width=\textwidth]{img/activityFitness}
\caption{Starten einer Fitnessapp durch den Benutzer.}\label{fig:activityFitness}
\end{figure}

\begin{figure}[h]
\centering\
\includegraphics[width=\textwidth]{img/activityAppKit}
\caption{Starten einer AppKit-Applikation durch den Benutzer.}\label{fig:activityAppKit}
\end{figure}

\begin{figure}[h]
\centering\
\includegraphics[width=\textwidth]{img/activityLaden}
\caption{Laden des SmartWatch-Armbandes über die mitgelieferte Induktionsplatte.}\label{fig:activityLaden}
\end{figure}

\section{Sequence Diagram}

Das folgende Sequenzdiagramm zeigt das Szenario einer erhaltenen Mitteilung. Zu Beginn erhält die Smartphone die Nachricht (,,message''). Die erhaltene Nachricht leitet er weiter an Smartwatch mit dem Operator sendMessage(). Danach wird die Nachricht am Display angezeigt, die Smartwatch schickt dem Display eine Nachricht mit showMessage() und zur selben Zeit kann eine Benachrichtigung stattfinden. Dabei gibt es vier Möglichkeiten einer Benachrichtigung.
1.: Die Bedingung ist, dass Vibration und Sound an ist. Die Smartwatch vibriert und anschließend kommt auch der Benachrichtigungston.
2.: Die Bedingung ist, dass die Vibration an ist und der Sound aus ist. Die Smartwatch vibriert, allerdings kommt kein Benachrichtigungston.
3.: Die Bedingung ist, dass die Vibration aus ist und der Sound an ist. Die Smartwatch vibriert nicht, stattdessen kommt ein Benachrichtigungston.
4.: else. Vibration und Sound sind aus.

\begin{figure}[h]
\centering\
\includegraphics[width=\textwidth]{img/seqMessage}
\caption{Sequenzdiagramm zum erhalten einer Nachricht.}\label{fig:seqMessage}
\end{figure}
