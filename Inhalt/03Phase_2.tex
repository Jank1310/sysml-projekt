\chapter{Phase 2: System Interaktion und Bedienung}

\section{User Stories}

Nun werden die Usecases aus Phase 1 mithilfe von User Stories weiter beschrieben. Dabei liegt der Fokus auf Szenarien, die eine Interaktion mit dem Smartwatch-System abbilden.

Die ersten beiden User Stories beschreiben zunächst einen typischen Hauptanwendungsfall der Uhr: Das Annehmen/Ablehnen/Tätigen von Anrufen von dem verbundenen Smartphone.

\begin{figure}[H]
\centering\
\includegraphics[width=14cm]{img/story_in}
\caption{User Story - Eingehender Anruf}\label{fig:story-in}
\end{figure}

\begin{figure}[H]
\centering\
\includegraphics[width=14cm]{img/story_out}
\caption{User Story - Anruf tätigen}\label{fig:story-out}
\end{figure}

Um dem Produkt eine detailgetreue Anleitung für einen reibungslosen Einrichtungsprozess beizulegen, wurde eine User Story speziell für das Einrichten, bzw. Koppeln der Smartwatch mit einem Smartphone definiert.
\begin{figure}[H]
\centering\
\includegraphics[width=14cm]{img/story_pairing}
\caption{User Story - Pairing}\label{fig:story-pairing}
\end{figure}

Da eine Kernfunktionion der Smartwatch darin besteht, sie für Fitness-Tracking verschiedener sportlichen Aktivitäten einzusetzen, wurden mehrere User Stories für diese Anwendungsfälle entworfen. Im folgenden wird die User Story "Joggen gehen" beschrieben, weitere Fitness-User Stories finden sich im Anhang an diese Arbeit.
\begin{figure}[H]
\centering\
\includegraphics[width=14cm]{img/story_joggen}
\caption{User Story - Fitness}\label{fig:story-joggen}
\end{figure}

\section{Sequence Diagram}

\section{State Diagram}

\section{Timing Diagram}

