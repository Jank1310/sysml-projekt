\chapter{Phase 1: Spezifikation und Grob-Design}

Text Phase 1 kommt hier...
\section{Use Case Diagram}

\section{Requirement Diagram}

\section{Package Diagram}

\section{Block Definition Diagram !!!Nochmal durchlesen!!!überarbeiten!!! auf die Values der Blocks eingehen!!!}
Das Blockdefinitionsdiagramm !!!Verweis auf Anhang!!! in dieser Phase stellt den groben, physikalischen Aufbau der Smartwatch dar. Diese setzt sich zusammen aus Sensoren, einem Display, einem Speichermedium, einem Armband, einen Gyroskop, einer Bluetooth-Antenne und einem GPS-Modul. Um die Stromversorgung der Uhr zu garantieren, soll das Armband aus mehreren Akkuzellen bestehen. Das Display soll dabei komplett von der Uhr trennbar sein. So kann die Uhr bei schwachen Akku oder bei einem Wechsel zu einer sportlichen Tätigkeit einfach in einem anderen Armband eingesetzt werden. Die Bluetooth-Antenne soll die Verbindungsstelle zum Smartphone sein. Um die Akkulaufzeit zu erhöhen soll die Antenne, zusätzlich zum normalen Bluetooth, auch "Low-Energy" -Bluetooth unterstützen. Die Smartwatch selbst soll keine Daten, außer die für die Nutzung reiner Sportaktivitäten benötigten, speichern können. Daher ist der Speicher recht klein gehalten und dient nicht zum verwalten von Bildern oder anderen Benutzerdaten. 
Zu diesem Zeitpunkt der Entwicklung waren die Funktionalitäten noch nicht fertig festgelegt und deshalb fehlen im Diagramm unter anderem die Tasten für das Skalieren und Weiterblättern des Bildschirms und eine genauere Spezifikation der Sensoren. 