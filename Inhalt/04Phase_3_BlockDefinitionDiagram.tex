\section{Block Definition Diagram}
In \textbf{Phase 3} stellt das \textbf{Block Definition Diagram} wieder den \textbf{physikalischen Aufbau} der \textit{SmartWatch} dar \ref{fig:block2}. Da nun alle \textbf{Funktionalitäten} fertig definiert wurden, ist dies der \textbf{endgültige Aufbau} der Uhr. Die zwei \textbf{Hauptkomponenten} der \textit{SmartWatch} sind weiterhin das \textit{SmartWatch-Modul} und das \textit{Armband}. Wobei das \textit{Modul} die \textbf{Funktionalitäten} und die \textbf{Software} verwaltet und das \textit{Armband} die Uhr mit Strom versorgt. Im Zentrum des Diagramms das \textit{SmartWatch-Modul}. Dieses besitzt eine \textit{Bluetooth-Antenne}. Über diese \textbf{Schnittstelle} kommuniziert die \textit{Uhr} mit einem gekoppelten \textit{SmartPhone}. Das \textit{Modul} besitzt einen \textit{Display} dessen Auflösung in Pixel angegeben wird. Dieser wird über \textit{drei} \textit{Tasten} bedient. \textit{Eine Taste} ist das \textit{Zoomrad}. Damit wird die \textbf{Schriftgröße} angezeigter Nachrichten oder laufender Applikationen gesteuert. Zusätzlich kann mit den \textit{Scrolltasten} die \textit{SmartWatch} aus dem \textbf{Ruhezustand} geholt oder nach dem \textbf{Herunterfahren} wieder gestartet werden. Eine weitere Komponente ist der interne \textit{Speicher} zum verwalten der \textbf{nativen Fitnessapplikationen} und \textbf{Fehlerprotokollen}. Außerdem hält das \textit{SmartWatch-Modul} einen \textit{Vibrationsmotor}. Auf diesen \textit{Motor} können \textbf{Applikationen} vom \textit{Smartphone} oder auch die \textbf{nativen Fitnessfunktionen} zugreifen, um den Benutzer über spezielle Ereignisse zu informieren. Für die \textbf{Tonausgabe} bei Verwendung einer \textbf{Musikapplikation} oder bei einem \textbf{Telefonat} ist ein \textit{Lautsprecher} vorhanden. Dieser gibt jede Form von Audiosignal, die sonst über die \textit{SmartPhone-Lautsprecher} ausgegeben werden würde, stattdessen über die \textit{SmartWatch} aus. Das integrierte \textit{Mikrophon} dient als \textbf{Kommunikationsschnittstelle} für Telefonate. Das aufgefangene \textbf{Audiosignal} wird an das \textit{Smartphone} weitergeleitet und dient als Ersatz für das \textit{Mikrophon} im \textit{SmartPhone}. Zur Verwendung durch die \textbf{nativen Fitnessapplikationen} und auch für den Zugriff durch andere \textbf{SmartPhone-Applikationen} stehen ein \textit{GPS-Modul} und \textit{Sensoren} zur Verfügung. Da die \textbf{Fitness-Funktionalität} auch ohne \textit{SmartPhone} ausführbar sein sollen, können dafür nicht die vorhandenen Modalitäten des \textit{SmartPhones} verwendet werden. Als \textit{Sensoren} stellt die \textit{SmartWatch} ein \textit{Gyroskop}, ein \textit{Temperatursensor} und ein \textit{Pulsmesser}. Das \textit{Gyroskop} ermöglicht es, in Verbindung mit dem \textit{GPS}, zurückgelegte Strecken zu messen. Der \textit{Temperatursensor} stellt !!!Markus fragen!!! die Umgebungstemperatur. Der letzte \textit{Sensor} ist der \textit{Pulsmesser}. Dieser misst den Puls des Benutzer über das Handgelenk.\\
Als nächste \textbf{Hauptkomponente} wird das \textit{Armband} betrachtet. Das \textit{Armband} soll die \textbf{Stromversorgung} der \textit{SmartWatch} regeln und ist austauschbar. So gibt es \textbf{verschiedene Modelle} für \textbf{Fitness} und \textbf{Alltag}. Diese halten eine unterschiedliche Anzahl an \textit{Akkuzellen} und bestehen deshalb aus verschiedenen Materialien. Aus der Verwendung unterschiedlicher Materialien folgt wiederum ein Unterschied in Gewicht und Komfort der \textit{Armbänder}.
\begin{figure}[h]
\centering\
\includegraphics[width=\textwidth]{img/block2}
\caption{Endgültiger physikalischer Aufbau der SmartWatch in Phase 3 anhand eines Block Definition Diagrams.}\label{fig:block2}
\end{figure}