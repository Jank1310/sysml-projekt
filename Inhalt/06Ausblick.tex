\chapter{Ausblick}

Osmans/Ahmets Text hier...
\\
\todo{Korrekturlesen}
Um die Effizienz in einem Projekt zu steigern, ist es notwendig, den Entwicklern effektive Notationen, Techniken und Methoden zur Verfügung zu stellen. Weil das primäre Ziel jeder Softwareentwicklung das lauffähige und korrekt implementierte Produktionssystem ist, sollte der Einsatz der UML nicht nur zur Dokumentation von Entwürfen dienen. Es sollte bei der Entwicklung zum Diskussion anregen. Da in der heutigen Zeit die Agile Vorgehensmodell stark verbreitet ist, so sollte man eine "Agile Modellierung" entwerfen. Bei der Agile Modellierung werden einige wesentliche Grundelemente agiler Softwareentwicklungsmethoden herausgearbeitet und auf Basis der dabei gewonnenen Erkenntnisse eine agile modellbasierte Entwicklungsmethode skizziert. Kern agiler Entwicklungsmethoden ist die Nutzung von Modellen als Darstellungs- und Diskussionsmittel, insbesondere aber auch zur Programmierung und Testfalldefinition durch Codegenerierung und zur Planung von Evolutionsschritten durch modellbasiertes Refactoring.