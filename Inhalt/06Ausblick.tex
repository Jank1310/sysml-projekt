\chapter{Ausblick}

Um die Effizienz in einem Projekt zu steigern, ist es notwendig, den Entwicklern effektive Notationen, Techniken und Methoden zur Verfügung zu stellen. Weil das primäre Ziel jeder Softwareentwicklung das lauffähige und korrekt implementierte Produktionssystem ist, sollte der Einsatz der UML nicht nur zur Dokumentation von Entwürfen dienen, sondern sollte bei der Entwicklung auch zur Diskussion anregen. Da in der heutigen Zeit das dynamische Vorgehensmodell stark verbreitet ist, ist es notwendig, dass das Designen der Software flexibel ist. Bei einer dynamischer Modellierung werden einige wesentliche Grundelemente flexiblen Softwareentwicklung herausgearbeitet und auf Basis der dabei gewonnenen Erkenntnisse eine dynamische und modellbasierte Entwicklungsmethode skizziert. Kern dynamischer Entwicklungsmethoden ist die Nutzung von Modellen als Darstellungs- und Diskussionsmittel, insbesondere aber auch zur Programmierung und Testfalldefinition durch Codegenerierung und zur Planung von Evolutionsschritten durch modellbasiertes Refactoring.