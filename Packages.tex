% Anpassung des Seitenlayouts --------------------------------------------------
%   siehe Seitenstil.tex
% ------------------------------------------------------------------------------
\usepackage[
    automark, % Kapitelangaben in Kopfzeile automatisch erstellen
    headsepline, % Trennlinie unter Kopfzeile
    ilines % Trennlinie linksbündig ausrichten
]{scrpage2}

% Sprache und Umlaute
\usepackage[ngerman]{babel}
\usepackage[utf8]{inputenc}
\usepackage[T1]{fontenc}
\usepackage{textcomp}

% Schrift ----------------------------------------------------------------------
\usepackage{lmodern}
\usepackage{relsize}

% Grafiken ---------------------------------------------------------------------
\usepackage[dvips,final]{graphicx}
\graphicspath{{img/}}

\usepackage{pdfpages}

% für Index-Ausgabe mit \printindex --------------------------------------------
\usepackage{makeidx}

% Einfache Definition der Zeilenabstände und Seitenränder etc. -----------------
\usepackage{setspace}
\usepackage{geometry}
\usepackage{dirtree}


% zum Einbinden von Programmcode -----------------------------------------------

% es wird das Texments Package zusammen mit Pygments verwendet
% Pygments bietet Syntax Highlighting für Swift (und andere Sprachen)
% Installationshinweise auf http://www.zoppelt.net/2014/08/using-swift-language-in-latex/

% \usepackage{texments}
% \usestyle{xcode}


% URL verlinken, lange URLs umbrechen etc. -------------------------------------
\usepackage{url}
\makeatletter
\makeatother

% wichtig für korrekte Zitierweise ---------------------------------------------
\usepackage{natbib}
% \usepackage[sort\&compress,numbers]{natbib}
% \usepackage{url}
% \usepackage[htt]{hyphenat}
% \usepackage{hyperxmp}

% PDF-Optionen -----------------------------------------------------------------
% fortlaufendes Durchnummerieren der Fußnoten ----------------------------------
\usepackage{chngcntr}

% Lange Tabellen ---------------------------------------------------------------
\usepackage{longtable}
\usepackage{array}
\usepackage{ragged2e}
\usepackage{lscape}

% Spaltendefinition rechtsbündig mit definierter Breite ------------------------
\newcolumntype{w}[1]{>{\raggedleft\hspace{0pt}}p{#1}}
\newcolumntype{L}[1]{>{\raggedright\let\newline\\\arraybackslash\hspace{0pt}}m{#1}}
\newcolumntype{C}[1]{>{\centering\let\newline\\\arraybackslash\hspace{0pt}}m{#1}}
\newcolumntype{R}[1]{>{\raggedleft\let\newline\\\arraybackslash\hspace{0pt}}m{#1}}

% Formatierung von Listen ändern -----------------------------------------------
\usepackage{paralist}

% bei der Definition eigener Befehle benötigt
\usepackage{ifthen}

% definiert u.a. die Befehle \todo und \listoftodos
\usepackage{todonotes}

% sorgt dafür, dass Leerzeichen hinter parameterlosen Makros nicht als Makroendezeichen interpretiert werden
\usepackage{xspace}
\usepackage[bottom]{footmisc}
\usepackage{units}
\usepackage[section]{placeins}
\usepackage{float}


\usepackage[
    bookmarks,
    bookmarksnumbered=true, % Zeigt nummerierung an
    bookmarksopen=true,		% Zeigt Lesezeichenleiste in PDF-Reader an
    bookmarksopenlevel=1, 	% gibt die Tiefe an
    colorlinks=true,
% diese Farbdefinitionen zeichnen Links im PDF farblich aus
      linkcolor=black, 	% einfache interne Verknüpfungen
      anchorcolor=black,	% Ankertext
      citecolor=blue, 	% Verweise auf Literaturverzeichniseinträge im Text
      filecolor=magenta, % Verknüpfungen, die lokale Dateien öffnen
      menucolor=blue, 	% Acrobat-Menüpunkte
      urlcolor=blue,
% diese Farbdefinitionen sollten für den Druck verwendet werden (alles schwarz)
%   linkcolor=black, % einfache interne Verknüpfungen
%   anchorcolor=black, % Ankertext
%   citecolor=black, % Verweise auf Literaturverzeichniseinträge im Text
%   filecolor=black, % Verknüpfungen, die lokale Dateien öffnen
%   menucolor=black, % Acrobat-Menüpunkte
%   urlcolor=black,
    backref,
    plainpages=false, 	% zur korrekten Erstellung der Bookmarks
    pdfpagelabels, 		% zur korrekten Erstellung der Bookmarks
    hypertexnames=false, % zur korrekten Erstellung der Bookmarks
    %linktocpage, 		% Seitenzahlen anstatt Text im Inhaltsverzeichnis verlinken
    %pdfborder={0 0 0},	% kein Rahmen
    breaklinks=true
]{hyperref}

% Befehle, die Umlaute ausgeben, führen zu Fehlern, wenn sie hyperref als Optionen übergeben werden
\hypersetup{
    pdftitle={\titel}, %\untertitel},
    % pdfauthor={\autor},
    % pdfcreator={\autor},
    pdfsubject={\titel}, %\untertitel},
    pdfkeywords={\titel}, %\untertitel},
}

\usepackage{appendix}
